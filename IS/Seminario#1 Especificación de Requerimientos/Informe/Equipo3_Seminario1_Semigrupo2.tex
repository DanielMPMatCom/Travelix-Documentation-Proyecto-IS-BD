\documentclass{article}
\usepackage[spanish]{babel}
\usepackage[utf8]{inputenc}
\usepackage{hyperref}
\usepackage{geometry}
\usepackage{graphicx}
\usepackage{pdfpages}
\usepackage{pdflscape}
\usepackage{svg}
\usepackage{fontawesome5}
\usepackage{subcaption}
\geometry{margin=2in}


\title{\huge{Especificación de Requerimientos de Software \\ para \\ Agencia de Viajes. \\ Travelix.}

\includegraphics[width=0.5\textwidth]{travelix.png}
}
\author{Daniel Toledo \and Daniel Machado \and Osvaldo Moreno \and José Antonio Concepción \\ Líder: Daniel Toledo \\ \\ Equipo 3 Semigrupo 2}
\date{\today}

\begin{document}

\maketitle
\thispagestyle{empty}
\newpage

\tableofcontents
\newpage

\section{Propósito del Documento}

\subsection{Alcance del Producto}
El alcance del producto incluye la creación de una aplicación que permita a las agencias 
de viajes gestionar eficientemente sus servicios, reservaciones y actividades relacionadas.
\section{Descripción General}

\subsection{Funciones del Producto}
    \textbf{Actualización de Ofertas:}
        Permite al personal del departamento de marketing actualizar ofertas, incluyendo 
        información sobre hoteles, excursiones y paquetes turísticos.

    \textbf{Reservaciones:}
        Facilita a los agentes de venta la realización de reservaciones para turistas 
        individuales y grupos, gestionando detalles como compañía aérea, hoteles, fechas, etc.

    \textbf{Consulta de Información:}
        Proporciona funciones para consultar información detallada, como el listado de hoteles 
        en paquetes, la cantidad de reservaciones, etc.

\subsection{Características de los usuarios}
    \textbf{Experiencia:}
        Se espera que el personal del departamento de marketing tenga experiencia en la actualización 
        de ofertas y que los agentes de venta tengan conocimientos básicos en la realización de 
        reservaciones. Los turistas individuales pueden tener variados niveles de experiencia en 
        el uso de aplicaciones de reservas en línea.

    \textbf{Plataformas:}
        La aplicación deberá ser amigable tanto para usuarios con experiencia en entornos de oficina 
        como para turistas individuales que pueden acceder desde diferentes dispositivos, 
        como computadoras y dispositivos móviles.

\subsection{Restricciones Generales}
    \textbf{Aspecto Visual:}
        Se prefiere que el diseño visual se asemeje a sitios web existentes, por ejemplo, adoptando 
        un estilo similar a Tripadvisor.

    
\subsection{Dependencias y suposiciones}
    \textbf{Navegación:}
    Se establece una restricción en la navegación, limitándola a no más de tres niveles de profundidad.

    \textbf{Ubicación de Menús:}
    Se sugiere que los menús fijos estén ubicados en el lateral derecho de la interfaz.


\section{Requerimientos Específicos}
\subsection{Requerimientos Funcionales}
    $\cdot$ \textbf{Obtención del Listado de Hoteles en Paquetes:}
        La aplicación deberá proporcionar una funcionalidad que permita a los usuarios 
        obtener de manera eficiente el listado completo de hoteles que están incluidos 
        en los paquetes de viajes ofrecidos por las agencias. Esta información es crucial 
        para los turistas que desean conocer las opciones de alojamiento disponibles en los 
        paquetes turísticos.

     $\cdot$ \textbf{Cálculo de Reservaciones y Monto Total:}
        La aplicación debe ser capaz de realizar un seguimiento preciso de todas las reservaciones 
        realizadas en las agencias de viajes. Deberá calcular automáticamente la cantidad total de 
        reservaciones realizadas y el monto económico total asociado. Esto proporcionará a las agencias
         información valiosa sobre la demanda de sus servicios y permitirá una mejor planificación y 
         gestión financiera.

    $\cdot$ \textbf{Obtención de Datos de Turistas Frecuentes:}
        La aplicación deberá identificar y presentar los nombres y direcciones electrónicas de 
        aquellos turistas que han realizado reservaciones individuales a Cuba en más de una ocasión. 
        Esto facilitará a las agencias de viajes establecer relaciones más personalizadas con estos clientes 
        frecuentes, ofreciendo servicios adaptados a sus preferencias.

    $\cdot$ \textbf{Obtención de Horarios y Lugares de Salida de Excursiones en Fines de Semana Extendidos:}
        La aplicación debe proporcionar una funcionalidad que permita obtener la información detallada 
        sobre los lugares y horarios de salida de todas las excursiones ofrecidas durante los fines de 
        semana extendidos (viernes, sábado y domingo). Esta información, ordenada de manera apropiada, 
        será presentada en la interfaz de la agencia para su consulta y visualización.

    $\cdot$ \textbf{Análisis de Paquetes con Precio Superior al Promedio:}
        La aplicación debe ser capaz de analizar y calcular la cantidad de paquetes cuyo precio está por encima 
        del precio promedio de todos los paquetes disponibles. Este análisis permitirá a las agencias identificar 
        y destacar paquetes de mayor valor, así como ajustar estrategias de precios.

    $\cdot$ \textbf{Reservación Apropiada para Turista Individual:}
        La aplicación deberá proporcionar una funcionalidad eficiente para realizar reservaciones que 
        correspondan de manera apropiada a un turista individual. Esto implica seleccionar la compañía 
        aérea, gestionar la reserva en los hoteles deseados, establecer fechas de llegada y salida, y calcular 
        el precio total de manera precisa.

    $\cdot$ \textbf{Aspectos generales del sistema: }
        El sistema debe ser "tipo web, para que cualquier persona pueda acceder a los servicios que se brindan 
        y los trabajadores también tengan una interacción sencilla". 
    
    $\cdot$ \textbf{Posibles aspectos futuros: }
        Dadas las interacciones entre cliente y operador de ventas, es posible que un futuro se quiera registrar 
        el operador que tramita la venta para obtener bonos específicos por cantidad de ventas.

\subsection{Requerimientos no funcionales}
    $\cdot$ \textbf{Eficiencia en el Rendimiento:}
        La aplicación debe mantener un rendimiento eficiente incluso cuando maneje grandes cantidades de datos, 
        garantizando tiempos de respuesta rápidos y una experiencia de usuario fluida.

    $\cdot$ \textbf{Interfaz de Usuario Intuitiva:}
        La interfaz de usuario debe ser intuitiva y fácil de usar para el personal del departamento de marketing 
        y los agentes de venta, sin requerir un extenso entrenamiento.

    $\cdot$ \textbf{Seguridad de Datos:}
        La seguridad de la información, especialmente los datos personales de los turistas, debe ser una prioridad. 
        La aplicación debe implementar medidas robustas de seguridad para proteger la privacidad y confidencialidad 
        de la información del cliente. Debe existir un proceso de autenticación básico para cada usuario. Además se 
        debe crear una jerarquía de permisos para diferenciar administradores de turistas. 

\subsection{Requerimientos de entorno}
    $\cdot$ \textbf{Compatibilidad con Sistemas Operativos:}
        La aplicación debe ser compatible con una variedad de sistemas operativos, incluyendo Windows, macOS y Linux, 
        para garantizar su accesibilidad a una amplia audiencia de usuarios.

    $\cdot$ \textbf{Conexión a Internet:}
        Para la actualización en tiempo real de ofertas y la realización de reservaciones, la aplicación debe requerir 
        acceso a una conexión a Internet estable y segura.

    $\cdot$ \textbf{Escalabilidad:}
        La aplicación debe ser diseñada de manera que sea escalable, capaz de manejar el crecimiento en la cantidad de 
        agencias, turistas y servicios ofrecidos sin comprometer su rendimiento.

    $\cdot$ \textbf{Tecnologías:}
        El cliente no tiene conocimientos avanzados de informática por lo que le proponemos dadas sus peticiones y amplia 
        capacidad adquisitiva: \\ 
        - Alquilar servidores para las bases de datos en la nube. \\
        - Utilizar React y .Net para el desarrollo de la app.
\newpage
\section{Anexos}

\end{document}
