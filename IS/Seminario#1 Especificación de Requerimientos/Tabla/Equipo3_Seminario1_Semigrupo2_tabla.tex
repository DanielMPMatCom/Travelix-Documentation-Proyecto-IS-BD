\documentclass[10pt,a4paper]{article}
\usepackage[spanish]{babel}
\usepackage[utf8]{inputenc}
\usepackage{multirow}
\usepackage{tabularx}
\usepackage[margin=1cm,left=2cm]{geometry}


\title{Evaluación de un proyecto de turismo}

\begin{document}

\begin{center}    
    \begin{table}[t]
        \begin{tabular}{|c|c|m{5cm}|}
            \cline{1-3}
            \multirow{3}{*}{\textbf{Características del Problema:}}     
            & Predecible                 
            &  
            \textbf{5} - El problema se reduce a la venta de paquetes variando en servicios que pueden ser 
            predecibles dentro del scope del turismo. Fácilmente podemos añadir nuevos paquetes, nuevas agencias, 
            podemos crear un apartado distinto para ventas de recuerdos turísticos.
            \\ \cline{2-3}
            
            & Repetible                             
            &
            \textbf{4} - El problema planteado es lo suficientemente genérico como para extrapolarlo a otros 
            negocios en la rama de la oferta de servicios.
            \\ \cline{2-3}
            
            & Complejidad                           
            & 
            \textbf{3} - Es un reto porque tenemos que enfrentarnos por primera vez a un proyecto 
            completo y tenemos que dominar nuevas tecnologías en un lapso de tiempo relativamente corto.
            
            \\ \cline{1-3}
            
            \multirow{3}{*}{\textbf{Características del Personal:}}      
            & Experiencia con el problema          
            & 
            \textbf{1} - Nunca hemos trabajado con el sector de servicios.
            \\ \cline{2-3}
            
            & Productividad                         
            & 
            \textbf{4} - El personal se ha esforzado en tareas previas por la excelencia de sus proyectos.  
            \\ \cline{2-3}
            
            & Jefe de equipo                        
            & Daniel Toledo, fue escogido por su experiencia en proyectos previos y actitud positiva 
            ante los desafíos.
            \\ \cline{1-3}

            \multirow{3}{*}{\textbf{Características del Cliente:}}       
            & Habilidades técnico-computacionales   
            &
            \textbf{3} - Por como fue ofrecida la información del problema se deduce que el cliente tiene 
            conocimientos básicos de gestión de bases de datos pero desconoce de la informatización del proceso.
            \\ \cline{2-3}
            
            & Disponibilidad                        
            &
            \textbf{5} - El cliente está dispuesto a atender todas nuestras inquitudes en cualquier horario.
            \\ \cline{2-3}
            & Claridad y precisión en lo que desea  
            & 
            \textbf{4} -  Tiene una visión clara de la aplicación que desea aunque está abierto a sugerencias y nuevos 
            enfoques.
            \\ \cline{1-3}

            \textbf{Tiempo: 616 horas totales, 8 al día}
            & \multicolumn{2}{l|}{\textbf{Esfuerzo: 11 hombres para terminar en un mes.}} 
            \\ \cline{1-3}
        \end{tabular}
    \end{table}
\end{center}

    \end{document}
    